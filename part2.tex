% вторая часть

\section{Разработка архитектуры микродрона}
\subsection{smthng}
Проведен анализ рынка радиоуправляемых квадрокоптеров и замечено, что готовых вариантов, подходящих для взаимодействия с разрабатываемой наземной станцией, нет. В связи с чем необходимо подобрать компоненты и собрать вручную.
Квадрокоптер состоит из:
-полетного контроллера
-4 регуляторов оборотов электродвигателей
-4 электродвигателей
-рамы
-камеры
-видеопередатчика
-видеоантенны
-радиоприемника
Набор наземной станции и квадрокоптера в основном планируется использовать в помещении. В случае использования БПЛА на улице, при весе свыше 250г требуется регистрация, согласно постановлению такому-то. Основываясь на этом, поставлены следующие условия к компонентам квадрокоптера:
-размер не должен превышать 140*140*50мм3
-полетный вес должен быть ниже 250г
-коптер должен выдерживать столкновения
-пропеллеры должны быть защищены
-минимальное полетное время 5 мин
-энергопотребление?
Подходя к вопросу выбора рамы, стоит учитывать такие факторы как:
-жесткость рамы
-легкий вес
-диагональную прочность
-стоимость
-расстояния между отверстиями, совпадающие с монтажными отверстиями на электронике

Были проведены испытания с рамами из разных материалов. Рассматривались следующие альтернативы: фанера, PLA и PETG пластики, текстолит, углепластик(карбон). Фанера обладает низкой стоимостью, но уступает по жесткости остальным альтернативам.(подумать насчет десернс таблицы). PLA пластик самый безопасный для здоровья человека, им можно печатать детали на 3d принтере, но не устойчив к ударам. PETG обладает большей прочностью по сравнению с ПЛА, но недостаточно жесткий, в связи с чем может вносить осцилляции в гироскоп, ухудшая работу ПИД регулятора полетного контроллера. Текстолит является самым жестким среди вышеперечисленных альтернатив, но обладает самым большим весом. Карбон уступает по стоимости, однако является самым жестким и относительно легким вариантом. Обладает достаточной прочностью, благодаря чему не вносит нежелательные осцилляции. Таким образом, было решено ставить карбоновую раму.
Защита для пропеллеров пластиковая, так как обладает упругостью и низкой стоимостью.

//критерии: жесткость(ударопрочность), прочность(гибкость/осцилляции), цена, вес, безопасность/вредность.
//альтернативы: фанера, пла, петг, текстолит, карбон

Форм фактор рамы также является немаловажной деталью. Для выполнения задач, где используется направление камеры вниз, вперед и вверх необходимо, чтобы защита пропеллеров, пластины рамы, а также аккумулятор не загромождали обзор. Оптимальным решением является рама с вытянутым корпусом и расположением лучей по типу deadcat (не могу не вставить это в презентацию)- передние лучи разведены на угол, близкий к 180 градусам. Расстояние между отверстиями для монтажа электроники выгоднее выбирать из стандартов - 16*16, 20*20 или 25,5*25,5 мм. Вариант 25,5*25,5мм рассматривать стоит только в том случае, если необходимо использовать плату, где разведены и полетный контроллер и регулятор. В этой работе такая плата неуместна, так как: в случае поломки заменяется полностью, стоимость больше, чем у стека из двух плат, и выбор такой платы с ресурсами, необходимыми для реализации моего проекта, крайне мал. Основываясь на вышеперечисленном была приобретена рама, представленная на рисунке 1.
Электроника квадрокоптера должна быть совместимой по характеристикам и габаритам. Регуляторы оборотов существуют 2х типов: раздельные для каждого мотора и размещенные на одной плате с монтажными отверстиями, как у полетного поктроллера. Для легкого БПЛА выгоднее ставить стек из платы с 4 регуляторами и полетного контроллера. 
Для управления с наземной станции полетный контроллер должен:
-обладать минимум 2 UARTами
-иметь процессор на базе F405/F745/F765 чипа

//Среди стеков с расстоянием между отверстиями 16*16 не оказалось варианта с регуляторами, способными выдавать 15А на каждый мотор




