
Идея проекта в создании образовательного робототехнического комплекта, состоящего из дрона и наземной станции. 

Суть проекта заключается в модификации и доработке существующих робототехнических решений на основе БПЛА посредством выноса бортового компьютера в наземную станцию, благодаря чему станет возможно уменьшение размеров и стоимости квадрокоптера.
Наземная станция представляет собой совокупность компьютера и радиомодулей, производит преобразование видеосигнала с борта дрона в координаты его положения в пространстве.

Таким образом, квадрокоптер получает локальную систему координат без использования датчиков подобных GPS, а также возможность навигации с использованием компьютерного зрения.
Разработанный ПАК дает возможность создать образовательный робототехнический комплект на основе беспилотника. Вынося бортовой компьютер в наземную станцию, получаем такие преимущества, как низкая стоимость, удешевление ремонта, возможность сделать дрон более компактным, чтобы повысить безопасность.

В ранее проделанной научно-исследовательской работе был спроектирован дрон для выполнения автономной миссии. Далее необходимо сконфигурировать наземную станцию и оборудование дрона для обмена данными, благодаря которым будет получена система координат для выполнения полетного задания.

Цель данной работы в конфигурации и реализации системы для отправки сообщений на борт дрона и получения данных с него.