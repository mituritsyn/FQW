\section{Глава 2. Конфигурация дрона для обнаружения местоположения}

%\https://discuss.ardupilot.org/t/indoor-autonomous-flight-with-arducopter-ros-and-aruco-boards-detection/34699

\subsection{Настройка ardupilot}
Прошила через qgroundcontrol на pixracer r15 ardupilot, произвела первоначальную настройку-выбрала тип рамы, откалибровала датчики.
Указала следующие параметры:
\begin{MyCode}
AHRS\_EKF\_TYPE 2
EKF2\_ENABLE 1
EKF3\_ENABLE 0
EK2\_GPS\_TYPE 3
EK2\_POSNE\_M\_NSE 0.1
EK2\_EXTNAV\_DELAY 80
GPS\_TYPE 0
COMPASS\_USE 0
VISO\_TYPE 0
\end{MyCode}

\subsection{Настройка Raspberry Pi zero w}

Для микрокомпьютера выбрана операционная система raspbian stretch lite.
%http://www.pcds.fi/downloads/operatingsystem/debianbased/raspbian/archive/stretch/raspbian.stretch.html
С помощью команд, представленных на листинге 1, записала образ на карту памяти микрокомпьютера.
\begin{MyCode}
\$ unzip -p 2018-11-13-raspbian-stretch-lite.zip
\$ sudo dd if=/home/qw/2018-11-13-raspbian-stretch-lite.img bs=4M of=/dev/sdd conv=fsync
\$ sync
\end{MyCode}
Для подключения к роутеру изменила параметры wpa\_supplicant.conf.
\begin{MyCode}
\$ less /etc/wpa_supplicant/wpa_supplicant.conf
ctrl_interface=DIR=/var/run/wpa_supplicant GROUP=netdev
update_config=1
country=RU

network={
	ssid="ИМЯ_ТОЧКИ_ДОСТУПА"
	psk=123456789
}
\end{MyCode}
%https://habr.com/ru/post/419947/
Для возможности удаленного подключения создан файл ssh в каталоге /boot.

После изменений перезагружается система и с помощью утилиты nmap на компьютере проверяются все подключения к роутеру:
\begin{MyCode}
\$ sudo nmap -sn 192.168.1.0/24
...
Nmap scan report for 192.168.1.148
Host is up (-0.062s latency).
MAC Address: B8:27:EB:D3:B7:09 (Raspberry Pi Foundation)
Nmap scan report for ikherty (192.168.1.28)
Host is up.
Nmap done: 256 IP addresses (4 hosts up) scanned in 4.42 seconds
...
\end{MyCode}
192.168.1.148-адрес Raspberry.
%https://www.raspberrypi.org/documentation/remote-access/ip-address.md
Для возможности запускать трансляцию видео с борта дрона необходимо установить следующие пакеты:
\begin{MyCode}
	gstreamer1.0
	gstreamer1.0-tools
	gstreamer1.0-plugins-good
	ser2net, после установки необходимо изменить /etc/ser2net.conf, добавив в конце строку, определяющую, как будет полетный контроллер общаться с микрокомпьютером
	2000:raw:0:/dev/ttyAMA0:115200 8DATABITS NONE 1STOPBIT
	и собрать gst-rpicamsrc из репозитория (https://github.com/thaytan/gst-rpicamsrc) с помощью команд make, make install; он необходим для использования rpi-камеры в качестве источника видеопотока во время трансляции.
\end{MyCode}
%https://discuss.ardupilot.org/t/indoor-autonomous-flight-with-arducopter-ros-and-aruco-boards-detection/34699/34

Информация о камере:
\begin{MyCode}
\$ v4l2-ctl --list-formats-ext -d /dev/video0
\end{MyCode}
%https://platypus-boats.readthedocs.io/en/latest/source/rpi/video/video-streaming-gstreamer.html

Уменьшили задержку с помощью передачи udp пакетов. Теперь она 150мс:
\begin{MyCode}
На ПК запускается:
\$ gst-launch-1.0 udpsrc port=5000 ! gdpdepay ! rtph264depay ! avdec_h264 ! videoconvert ! autovideosink sync=false

На Raspberry:
\$ gst-launch-1.0 rpicamsrc bitrate=1000000 ! 'video/x-h264,width=640,height=480' ! h264parse ! queue ! rtph264pay config-interval=1 pt=96 ! gdppay ! udpsink host=[IP ПК] port=5000
%\$ gst-launch-1.0 -v rpicamsrc bitrate=10000000 rotation=180 exposure-mode=10 awb-mode=0 awb-gain-red=1 awb-gain-blue=2 iso=800 shutter-speed=10000 contrast=50 ! "image/jpeg,width=640,height=480,framerate=30/1" ! udpsink host=192.168.1.148 port=9000
%https://www.raspberrypi.org/forums/viewtopic.php?t=196176
\end{MyCode}


%https://docs.opencv.org/master/db/da9/tutorial_aruco_board_detection.html
%https://diydrones.com/group/voltarobots/forum/connect-telemetry-through-tcp-udp?commentId=7447824%3AComment%3A1590641
%https://stackoverflow.com/questions/7669240/webcam-streaming-using-gstreamer-over-udp

%Для использования аппаратного кодирования:
%\begin{MyCode}
%PC(autovideosink sync=false not needed for gscam):
%gst-launch-1.0 udpsrc port=5000 ! gdpdepay ! rtph264depay ! avdec\_h264 ! videoconvert ! autovideosink sync=false

%RPI stream:

%gst-launch-1.0 rpicamsrc bitrate=1000000 ! "video/x-raw,width=640,height=480,framerate=30/1" ! omxh264enc target-bitrate=1000000 control-rate=variable ! 'video/x-h264,width=640,height=480'! h264parse ! queue ! rtph264pay config-interval=1 pt=96 ! gdppay ! udpsink host=192.168.1.253 port=5000
%\end{MyCode}