В современном мире технологии не стоят на месте, процессы постепенно автоматизируются, техника уменьшается в размерах, а способы управления машинами упрощаются. В программу основного общего образования включают курсы программирования и робототехники, закупаются образовательные комплекты.\footnote{Включение робототехники в программу основного общего образования как в виде занятий в рамках внеурочной деятельности, так и в качестве составной части школьного курса технологии полностью соответствует идеям, заложенным в федеральный государственный образовательный стандарт (ФГОС). Согласно приказу Минобрнауки России об утверждении ФГОС ООО, "при итоговом оценивании результатов освоения обучающимися основной образовательной программы основного общего образования должны учитываться сформированность умений выполнения проектной деятельности и способность к решению учебно-практических и учебно-познавательных задач" \cite{minobr}}
На рынке труда появляется спрос на специалистов в области БПЛА. БПЛА -- беспилотные летательные аппараты, используемые для мониторинга, перевозки грузов, образовательных проектов. В 2020 году существует счетное количество решений, способных удовлетворить спрос образовательных учреждений для обучения программированию, сборке и управлению беспилотниками. Но у всех имеются недостатки, такие как высокая стоимость, травмоопасность, малые вычислительные мощности. В связи с этим появилась идея создания такого комплекта, удовлетворяющего большинству требований и исключающего вышеперечисленные недостатки. Идея состоит в том, чтобы вынести за борт квадрокоптера вычислительную технику, тем самым уменьшив размеры дрона и возможные риски.

Таким образом, получается программно -- аппаратный комплекс из миниквадрокоптера и наземной управляющей станции. В данной научно -- исследовательской работе будет рассмотрена аппаратная часть такого комплекта, а также описаны протоколы взаимодействия наземной станции с квадрокоптером.
