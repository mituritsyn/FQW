\section{Разработка системы вычисления и передачи данных о положении дрона}

Отправка видеопотока производится посредством UDP протокола. В связи со спецификой отправляемых данных в реджиме онлайн 
Выбор протокола обоснован спецификой передаваемых данных: видеотрансляция в режиме реального времени, потому нет смысла использовать TCP, чтобы не выполнять пересыл отправленных пакетов.
% дописать
Данный протокол не гарантирует доставку сообщений, но позволяет уменьшить задержку получения потока, что более критично. Проверены кодеки jpeg, h264, omxh264enc, из них самым стабильным по задержке оказался h264.
% дописать особенности 1-программный кодек, 2-программный, 3-аппаратный возможнов теории описать кодеки
%так как система критична я к задержке видеосигнала, надо использовать систему сжатия данных- кодек. из доступных кодеков можно выделить перечисление.. кодек-программный алгоритм сжатия видео, но он настока часто юзается в современных системах что его поддержку реализуют на аппаратном уровне во многих современных процессорах. в нашей системе есть программная реализация кодека h264. к сожалению, пока не получилось добиться его стабильной работы, частично это связано с архитектурными особенностями и закрытостью видеоподсистемы рпи. (линк на архитектуру)
%посмотреть статью рпи архитектуры с электрички от вс\\

 Задержка составила 150 мс.

После включения дрона и загрузки ОС RPi производится удаленное подключение к Raspberry Pi.
Запускается gstreamer для передачи изображения с rpicamsrc на адрес наземной станции UDP пакетами. Команда представлена в листинге \ref{lst:12}:
\begin{Program}[H]
	\caption{Команда запуска gstreamer} \label{lst:12}
\begin{MyCode}
$ gst-launch-1.0 rpicamsrc bitrate=1000000 !\
'video/x-h264,width=640,height=480' ! h264parse !\
queue ! rtph264pay config-interval=1 pt=96 ! gdppay !\
udpsink host=[IP ПК] port=5000
\end{MyCode}
\end{Program}
% дописать можно указать что решение очень трепетнок пипелайну и в ходе нир очень много вркемени ушло на подборку параметров
На наземной станции в первой вкладке консоли запускается окружение mavros (листинг\ref{lst:13} ):
\begin{Program}[H]
\caption{Команда запуска окружения mavros} \label{lst:13}
\begin{MyCode}
$ roslaunch mavros px4.launch
\end{MyCode}
\end{Program}

После того, как произошли подключение и инициализация дрона в окружении, во второй	 вкладке консоли производится сброс координат дрона (листинг \ref{lst:14}):
\begin{Program}[H]
	\caption{Команда сброса координат дрона} \label{lst:14}
\begin{MyCode}
# инициализация параметров SET_GPS_GLOBAL_ORIGIN и SET_HOME_POSITION
$ rosrun aruco_gridboard set_origin.py
\end{MyCode}
\end{Program}

В третьей вкладке консоли запускается aruco\_gridboard, совмещенный с нодой gscam, для получения и обработки изображения (листинг \ref{lst:15}):
\begin{Program}[H]
\caption{Команды запуска aruco\_gridboard с нодой gscam} \label{lst:15}
\begin{MyCode}
$ export GSCAM_CONFIG="udpsrc port=5000 ! gdpdepay !\
rtph264depay ! avdec_h264 ! videoconvert"
$ roslaunch aruco_gridboard detection_rpicam.launch 
\end{MyCode}
\end{Program}

Для проверки корректности работы системы можно запустить программу rviz (листинг \ref{lst:16}):
\begin{Program}[H]
\caption{Команда запуска rviz} \label{lst:16}
\begin{MyCode}
$ rosrun rviz rviz -d catkin_ws/src/aruco_gridboard/data/aruco_grid.rviz
\end{MyCode}
\end{Program}

На рисунке \ref{fig:px4} отображены вывод в консоль координат, полученных aru\-co\_g\-rid\-board, изображение с камеры дрона с началом координат на карте маркеров и поле rviz с моделью дрона относительно карты.
% ~\ref{fig:px4}
\begin{figure}[H]
	\centering
	\includegraphics[width=0.7\linewidth]{pics/px4}
	\caption{ Скриншот экрана наземной станции во время определения положения дрона
	}
	\label{fig:px4}
\end{figure}

