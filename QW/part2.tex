% вторая часть

\section{Глава 2. Разработка архитектур}
\subsection{Разработка архитектуры микродрона}
\subsection{Компоненты квадрокоптера}

Набор наземной станции и квадрокоптера в основном планируется использовать в помещении. В случае использования БПЛА на улице, при весе свыше 250г требуется регистрация, согласно воздушному кодексу Российской Федерации \cite{ivp}. Основываясь на этом, поставлены следующие условия к компонентам квадрокоптера:

--- размер не должен превышать 140*140*50 \(мм^3\);

--- полетный вес должен быть ниже 250 г;

--- квадрокоптер должен выдерживать столкновения;

--- пропеллеры должны быть защищены;

--- обеспечена безопасность для детей;

--- минимальное полетное время 5 мин;

--- возможность обмениваться телеметрией.

В ходе проведения анализа рынка радиоуправляемых квадрокоптеров было выявлено, что готовых вариантов, соответствующих вышеперечисленным условиям, нет. В связи с чем необходимо подобрать компоненты и собрать вручную.

Подходя к вопросу выбора рамы, стоит учитывать такие факторы как:

--- прочность рамы;

--- легкий вес;

--- диагональную жесткость;

--- стоимость;

--- расстояния между отверстиями, совпадающие с монтажными отверстиями на электронике.

Диагональная жесткость важна для уменьшения собственной частоты колебаний рамы. Чем меньше собственная частота колебаний, тем больше фильтрации требуется, чтобы не вносить в гироскоп осцилляции, которые ухудшают работу ПИД регулятора полетного контроллера. Использование излишней фильтрации приведет к ПИД осцилляциям.

Были проведены испытания с рамами из разных материалов. Рассматривались следующие альтернативы: фанера, PLA, PETG и угольно -- армированный пластики, текстолит и углепластик (композитный пластик, также известный как карбон). Фанера обладает низкой стоимостью, но уступает по жесткости остальным альтернативам. PLA пластик самый безопасный для здоровья человека, им можно печатать детали на 3d принтере, но не устойчив к ударам. PETG обладает большей прочностью по сравнению с PLA, но недостаточно жесткий, в связи с чем уменьшается собственная частота колебаний. Угольно-армированный пластик позволяет обеспечить жесткость и прочность рамы, но является одним из самых дорогих вариантов и обусловлен трудностями печати.
Текстолит является самым жестким среди вышеперечисленных альтернатив, но обладает самым большим весом. Композитный пластик на основе карбонового волокна самый дорогой из перечисленных, однако является самым прочным, жестким и относительно легким вариантом. Таким образом, было решено использовать карбоновую раму.
Защита для пропеллеров пластиковая, так как обладает упругостью и низкой стоимостью.

Форм фактор рамы также является немаловажной деталью. Для выполнения задач позиционирования и навигации в зависимости от условий необходимо будет поворачивать камеру вниз, вперед и вверх. Исходя из этого, необходимо, чтобы защита пропеллеров, пластины рамы, а также аккумулятор не перекрывали обзор/уменьшали область видимости. Оптимальным решением является рама с вытянутым корпусом и расположением лучей по типу deadcat -- передние лучи разведены на угол, близкий к 180 градусам. Расстояние между отверстиями для монтажа электроники выгоднее выбирать из стандартов -- 16*16, 20*20 или 25,5*25,5 мм. Вариант 25,5*25,5мм рассматривать стоит только в том случае, если необходимо использовать “все в одном”: плату, совмещающую полетный контроллер и регуляторы в одном устройстве. Для поставленной цели -- создания учебного набора квадрокоптера такая плата неуместна по следующим причинам:

--- в случае поломки заменяется полностью;

--- стоимость выше, чем у комплекта раздельных регуляторов и полетного контроллера;

--- выбор такого формата плат, с ресурсами, необходимыми для реализации проекта, крайне мал.

Основываясь на вышеперечисленном была приобретена рама, представленная на рисунке \ref{fig:frame}. Она позволяет установить нано камеру (размером 14*14 мм), стеки из полетного контроллера и регуляторов с посадочными отверстиями 20x20mm/16x16mm, моторы размера 1102-1308 и пропеллеры диаметром до 40 мм.

\begin{figure}[H]
	\centering
	\includegraphics[width=0.5\linewidth]{../RW/pics/frame}
	\caption{Рама для экспериментального образца квадрокоптера
	}
	\label{fig:frame}
\end{figure}

Данная рама используется для создания экспериментального образца. В случае массового производства комплектов, которые будут получены при достижении поставленной цели, рама может быть заменена собственной разработкой.

Перейдем к выбору электроники.

Электроника квадрокоптера должна быть совместимой по характеристикам и габаритам. 

Для управления с наземной станции полетный контроллер должен:

--- обладать минимум 2 UART портами;

--- иметь процессор на базе F405 / F745 / F765 чипа.

UART (Universal asynchronous receiver/transmitter) -- это аппаратный последовательный интерфейс, который позволяет подключать датчики и периферию к полетному контроллеру. У него есть два вывода для внешнего соединения: TX -- для передачи данных, RX -- для приема.

UART порты потребуются для подключения устройства приема -- передачи телеметрии и возможности подключения дополнительной периферии.

Выбор чипа процессора основан на требованиях к ресурсам по памяти, производительности и периферии. Для того, чтобы прошить PX4, необходим объем памяти процессора не ниже 1 МБ. Такое условие выполняют процессоры на базе F405 / F745 / F765. Преимущество F7 чипов в том, что обеспечивается больше памяти и портов, а также лучше поддерживаются. Но они дороже, выбор полетных контроллеров на таких чипах меньше, а разработка собственного полетного контроллера пока не целесообразна.

Винто -- моторная группа должна быть оптимизирована под задачи автономного полета в помещении на небольшой скорости и устанавливаться в выбранную раму. У моторов бесколлекторного типа основными параметрами являются размеры статора -- неподвижной части мотора (4 цифры) и количество оборотов на вольт(kv). В четырехзначном числе первые два отвечают за диаметр статора, вторые -- за высоту статора. При одинаковых объемах статора крутящий момент на низких оборотах будет больше у того мотора, где больше диаметр статора, а на высоких оборотах там, где больше высота. Для экспериментального образца оптимальным выбором являются моторы 1202. Количество оборотов на вольт выберем, учитывая напряжение аккумулятора. Чем больше напряжение, тем меньше количество оборотов на вольт должно быть на моторе. Каждая ячейка, подключенная последовательно увеличивает напряжение на 4.2 В в заряженном состоянии. Для квадрокоптера с диагональю рамы 120 мм по соотношению вес / токоотдача наиболее выгодно ставить аккумуляторы с 2-3 ячейками. Основываясь на таблице характеристик, приведенных производителем, были выбраны моторы с 6000 kv (рис. \ref{fig:motor}).
Учитывая потребление тока моторами на полном газу и добавляя 10 -- 15 \% запаса, получаем характеристику регуляторов -- максимальный ток, проходящий через них. На экспериментальном образце он равен 15 А.
\begin{figure}[H]
	\centering
	\includegraphics[width=0.5\linewidth]{../RW/pics/motor}
	\caption{Моторы для экспериментального образца квадрокоптера
	}
	\label{fig:motor} % эта метка позволяет ссылаться на рисунок в тексте
\end{figure}
\begin{figure}[H]
	\centering
	\includegraphics[width=0.5\linewidth]{../RW/pics/stack}
	\caption{Стек электроники для экспериментального образца квадрокоптера
	}
	\label{fig:stack} % эта метка позволяет ссылаться на рисунок в тексте
\end{figure}
Видеопередатчик и камера выбирались исходя из поставленных условий. Видеосигнал аналогового типа дешевле и передается с задержкой меньше, чем цифровой сигнал. MVP решение будем основывать на аналоговом сигнале. Так как необходимо будет передавать видеопоток, камера должна иметь максимально возможное количество телевизионных линий -- разрешающая способность (TVL). Для камер нано формата это 1000 TVL. Размер изображения может быть как 4:3 и 16:9. Формат PAL / NTSC также может быть выбран на усмотрение.

Видеопередатчик обладает такими характеристиками как:

--- выходная мощность;

--- выходная частота;

--- количество каналов.

Для помещений мощность 25mW является оптимальной. Количество каналов должно быть выбрано таким образом, чтобы в случае совместных полетов сигнал не пересекался с сигналом другого беспилотника. Современные видеопередатчики имеют 40 каналов. Частота видеосигнала будет использоваться 5.8ГГц.

Для общения с наземной станцией квадрокоптеру понадобятся устройства приема-передачи телеметрии. Протокол, бод-рейт (скорость передачи данных для подключенного устройства приема-передачи данных) и частота устройств станции и квадрокоптера должны совпадать.

Программная часть квадрокоптера полностью осуществляется прошивкой PX4.

%//estimator lpe / ekf
%\url{https://dev.px4.io/v1.9.0/en/ros/offboard\_control.html}
%\url{https://dev.px4.io/v1.9.0/en/ros/external_position_estimation.html}
\subsection{Разработка архитектуры наземной станции}
Аппаратная часть наземной станции состоит из:

--- компьютера;

--- передающего модуля радиоуправления;

--- видеоприемника;

--- устройства приема-передачи телеметрии.

Наземная станция должна обмениваться телеметрией с квадрокоптером, получать видеопоток с квадрокоптера, и отправлять управляющий сигнал в виде команд MAVROS. Чем больше бод -- рейт подключенных модулей и меньше задержка сигнала, тем быстрее осуществляется выполнение команд. Для устройств приема-передачи телеметрии рекомендуется бод -- рейт, равный 921600. Учитывая эти факторы, выбираются устройства телеметрии и видеоприемник.
К компьютеру по UART порту подключается модуль радиоуправления и устройство приема-передачи телеметрии. Через USB порт подключается видеоприемник. Настраиваем видеоприемник на диапазон частот, соответствующий частотам видеопередатчика квадрокоптера и переходим к программной части.

Для позиционирования робототехнических систем с помощью компьютерного зрения используются aruco маркеры -- квадратные маркеры, состоящий из широкой черной границы и внутренней двоичной матрицы, которая определяет его идентификатор (id). Черная рамка облегчает ее быстрое обнаружение на изображении, а двоичная кодификация позволяет ее идентифицировать \cite{opencv}.

В ходе исследования был найден пакет aruco\_pose, предназначенный для работы с aruco-маркерами. Для распознавания маркеров разработан aruco-detect. Они входят в образ clover, который предоставляет пакеты и инструменты для позиционирования и управления квадрокоптером на базе ROS по MAVROS. Образ clover подходит для выполнения поставленной задачи.

Программная часть представляет собой совокупность взаимодействущего программного обеспечения, включающего в себя:

--- операционную систему на базе ядра linux;

--- пакет clover;

--- qgroundcontrol.

В qgroundcontrol выставляется бод -- рейт. Далее происходит подключение и обмен данными с устройством приема-передачи телеметрии, расположенного на борту квадрокоптера.
По MAVLink протоколу передаются данные в указанный UART, и все программы, прослушивающие этот UART, порт имеют доступ к данным с борта квадрокоптера. Для взаимодействия через ROS инструменты необходима настройка всех параметров подключения. На данном этапе НИР используется готовой решение от Copter Express -- пакет clover, позволяющий производить настройку максимально просто. В launch файлах пакета прописываются все параметры. Указываем UART, бауд -- рейт, и в консоли запускаем clover с помощью команд:
\begin{MyCode}
	gs@groundstation:~$ source /home/clover/catkin\_ws/devel/setup.bash
	gs@groundstation:~$ roslaunch clover clover.launch
\end{MyCode}

После этого доступны все инструменты clover. На листинге \ref{lst:5} приведен результат выполнения команды для получения телеметрии. "frame\_id: ''" означает, что показания берутся в системе координат относительно квадрокоптера.
\begin{Program}[H]
	\caption{Вывод телеметрии квадрокоптера в консоли} \label{lst:5}
	\begin{MyCode}
		gs@groundstation:~$ rosservice call /get_telemetry "frame_id: ''" 
		frame_id: "map"
		connected: True
		armed: False
		mode: "MANUAL"
		x: -0.00260298536159
		y: -6.72723326716e-05
		z: 0.00103790743742
		lat: 0.0
		lon: 0.0
		alt: 0.0
		vx: -0.00717502878979
		vy: -0.00176917202771
		vz: 0.00364218326285
		pitch: 0.0221049506217
		roll: -0.0172985047102
		yaw: 0.000302107335301
		pitch_rate: 0.00245076417923
		roll_rate: 0.00449034944177
		yaw_rate: 0.00266480189748
		voltage: 12.1499996185
		cell_voltage: 4.05000019073
		gs@groundstation:~$
	\end{MyCode}
\end{Program}

Видеоприемник подключен в порт /dev/video0. Указываем его в launch-файле, перезагружаем clover и проверяем в web-браузере топики по адресу localhost:8080 (рис. \ref{fig:topic}).

\begin{figure}[H]
	\centering
	\includegraphics[width=0.5\linewidth]{../RW/pics/topic}
	\caption{Список топиков, доступный по умолчанию
	}
	\label{fig:topic}
\end{figure}

Список НОД также меняется в launch файлах. В image raw топике будет отображаться видеопоток, полученный видеоприемником с борта квадрокоптера, в остальных топиках публикуются:

--- aruco-detect (на image raw определяются с помощью openCV aruco маркеры);

--- optical flow (на изображение с image raw накладывается точка отсчета для лазерного дальномера);

--- aruco map (карта маркеров, прописанная в конфигурационном файле).

Optical flow отключаем, так как на БПЛА не установлен лазерный дальномер.